\documentclass{article}

% formato
\usepackage[margin = 1.5cm, letterpaper]{geometry}
\usepackage[utf8]{inputenc}
\usepackage{graphicx}% http://ctan.org/pkg/graphicx
\usepackage{array}
\usepackage[table,xcdraw]{xcolor}

% árboles de dereivación
\usepackage{ebproof}

% autómatas
\usepackage{tikz}
\usetikzlibrary{automata, positioning}

%formato ecuaciones
\usepackage{amsmath}

% símbolos
\usepackage{amssymb}

% manejo de tablas
\usepackage{float}

\begin{document}
    \title{
        Lenguajes de Programación \\
        Ejercicio Semanal 6
    }

    \author{
        Sandra del Mar Soto Corderi \\
        Edgar Quiroz Castañeda
    }

    \date{
        20 de septiembre del 2019
    }
    
    \maketitle

    \begin{enumerate}
        \item {
            Utilizando el operador de punto fijo 
            $Y =_{def} \lambda f.(\lambda x.f(xx))(\lambda x.f(xx))$ responde 
            los siguientes incisos.
            
            \begin{enumerate}
            %a)
           
            	\item {
                    Define las funciones exponenciación de naturales \texttt{exp}
                    y reversa de listas \texttt{rev}.
                    
                    El cuerpo de la primera función sería algo como
                    \[
                        exp' := \lambda exp. \lambda x. \lambda y. 
                        if (iszero \ y) then \ 1 \ else \ x (exp \ x (pred \ y))
                    \]

                    Que con el operador $Y$ quedaría como $exp = Y exp'$

                    Para la reversa, se podría definir como sigue

                     \[
                         rev ' := \lambda rev. \lambda lr. \lambda l.
                         if (isnil \ l) \ then \ rl \ else \ (rev \ (cons \ (head \ l) \ rl)
                         \ (tail \ l) \ 
                         )
                     \]

                    Que es una funciónde recursiónde cola, donde $rl$ es el
                    acumulador que guarda parcialmente la lista bloqueada.

                    Entonces la función final sería $rev = (Yrev ') \ nil$
        		
        		}
        	
        	%b)
        	\item {
        		Encuentra la forma normal de las expresiones:
        		
        		\begin{itemize}
        			\item {
        			 \texttt{exp 2 2}
        			 
        			}
        		
        			\item {
        			\texttt{rev (cons 1 (cons 2 (cons 3 (cons 4 nil))))}
        			}
        		\end{itemize}
    		}
    	 \end{enumerate}
        }
    \end{enumerate}
\end{document}